% Options for packages loaded elsewhere
\PassOptionsToPackage{unicode}{hyperref}
\PassOptionsToPackage{hyphens}{url}
%
\documentclass[
]{article}
\usepackage{amsmath,amssymb}
\usepackage{iftex}
\ifPDFTeX
  \usepackage[T1]{fontenc}
  \usepackage[utf8]{inputenc}
  \usepackage{textcomp} % provide euro and other symbols
\else % if luatex or xetex
  \usepackage{unicode-math} % this also loads fontspec
  \defaultfontfeatures{Scale=MatchLowercase}
  \defaultfontfeatures[\rmfamily]{Ligatures=TeX,Scale=1}
\fi
\usepackage{lmodern}
\ifPDFTeX\else
  % xetex/luatex font selection
\fi
% Use upquote if available, for straight quotes in verbatim environments
\IfFileExists{upquote.sty}{\usepackage{upquote}}{}
\IfFileExists{microtype.sty}{% use microtype if available
  \usepackage[]{microtype}
  \UseMicrotypeSet[protrusion]{basicmath} % disable protrusion for tt fonts
}{}
\makeatletter
\@ifundefined{KOMAClassName}{% if non-KOMA class
  \IfFileExists{parskip.sty}{%
    \usepackage{parskip}
  }{% else
    \setlength{\parindent}{0pt}
    \setlength{\parskip}{6pt plus 2pt minus 1pt}}
}{% if KOMA class
  \KOMAoptions{parskip=half}}
\makeatother
\usepackage{xcolor}
\setlength{\emergencystretch}{3em} % prevent overfull lines
\providecommand{\tightlist}{%
  \setlength{\itemsep}{0pt}\setlength{\parskip}{0pt}}
\setcounter{secnumdepth}{-\maxdimen} % remove section numbering
\ifLuaTeX
  \usepackage{selnolig}  % disable illegal ligatures
\fi
\ifPDFTeX
  \TeXXeTstate=1
  \newcommand{\RL}[1]{\beginR #1\endR}
  \newcommand{\LR}[1]{\beginL #1\endL}
  \newenvironment{RTL}{\beginR}{\endR}
  \newenvironment{LTR}{\beginL}{\endL}
\fi
\IfFileExists{bookmark.sty}{\usepackage{bookmark}}{\usepackage{hyperref}}
\IfFileExists{xurl.sty}{\usepackage{xurl}}{} % add URL line breaks if available
\urlstyle{same}
\hypersetup{
  pdftitle={caracterización-conjunto-pr},
  hidelinks,
  pdfcreator={LaTeX via pandoc}}

\title{caracterización-conjunto-pr}
\author{}
\date{}

\begin{document}
\maketitle

\begin{quote}
Un conjunto S es {}-pr {} es el dominio de alguna función {}-pr. (Solo
caso composición)
\end{quote}

\begin{center}\rule{0.5\linewidth}{0.5pt}\end{center}

\hypertarget{pr--pr}{%
\subsection{\texorpdfstring{( {}-pr {}-pr
)}{( -pr -pr )}}\label{pr--pr}}

\begin{quote}
Predecesor de la característica. Solo va a estar definido para retorno
1.
\end{quote}

Tomemos la función {}.\\
Claramente {}.

\hypertarget{pr--pr-1}{%
\subsection{\texorpdfstring{( {}-pr {}-pr
)}{( -pr -pr )}}\label{pr--pr-1}}

Probaremos por inducción en k que {} es {}-pr para cada {}

\hypertarget{caso-0}{%
\subsubsection{(Caso 0)}\label{caso-0}}

\begin{quote}
Trivial
\end{quote}

Tenemos las siguientes funciones en {}

\begin{itemize}
\tightlist
\item
  {}
\item
  {}
\item
  {}
\item
  {}
\item
  {}
\item
  {}
\item
  {}
\end{itemize}

Y es trivial ver que todos sus dominios ({}) son {}-pr.

\hypertarget{caso-k1}{%
\subsubsection{(Caso k+1)}\label{caso-k1}}

\begin{quote}
Buscamos definir la característica de la composición en base a los
dominios de las funciones.\\
Clave 0: Por hipótesis todas las funciones y sus dominios van a ser
{}-pr.\\
Clave 1: Toda función {} {}-pr no total es la restricción de otra {}
{}-total\\
Clave 2: Definimos el conjunto S {}-pr como la intersección de los
dominios. Es decir que tenemos su característica.\\
Usamos las claves para definir la función
\end{quote}

óó

\textbackslash begin\{align\}\\
g
\&:D\emph{g\textbackslash subseteq\textbackslash omega\^{}n\textbackslash times\textbackslash Sigma\^{}\{\emph{m\}\textbackslash to
O,\textbackslash{}
O\textbackslash in\{\textbackslash omega,\textbackslash Sigma\^{}}\}
\textbackslash{}\\
g\_i
\&:D}\{g\emph{i\}\textbackslash subseteq\textbackslash omega\^{}l\textbackslash times\textbackslash Sigma\^{}\{*k\}\textbackslash to\textbackslash omega\textbackslash{}
,\textbackslash{} i=1,\textbackslash dots,n \textbackslash{}\\
g\_i
\&:D}\{g\_i\}\textbackslash subseteq\textbackslash omega\^{}l\textbackslash times\textbackslash Sigma\^{}\{\emph{k\}\textbackslash to\textbackslash Sigma\^{}}\textbackslash{}
,\textbackslash{} i=n+1,\textbackslash dots,n+m \textbackslash{}\\
\textbackslash end\{align\}

ó

S=\textbackslash bigcap\^{}\{n+m\}\emph{\{i=1\} D}\{g\_i\}

é

\textbackslash chi\emph{\{D\_F\}\^{}\{\textbackslash omega\^{}n\textbackslash times\textbackslash Sigma\^{}\{*m\}\}=\\
\textbackslash chi}\{D\emph{g\}\^{}\{\textbackslash omega\^{}n\textbackslash times\textbackslash Sigma\^{}\{*m\}\}\textbackslash circ\\
{[}\textbackslash bar g\_1,\textbackslash dots,\textbackslash bar
g}\{n+m\}{]}\textbackslash land\\
\textbackslash chi\_\{S\}\^{}\{\textbackslash omega\^{}n\textbackslash times\textbackslash Sigma\^{}\{*m\}\}

\end{document}
